
\chapter*{Abstract}
\addcontentsline{toc}{chapter}{Abstract}

This thesis presents the design and development of the supercapacitor assisted ``instant" water heating system as well as a new power converter topology to fast charge a supercapacitor bank.

 Delayed delivery of hot water in domestic water heating systems waste over 15 million cubic metres of treated water per annum in New Zealand. The patent pending supercapacitor assisted temperature modification apparatus~(SCATMA), solves this problem by using pre-stored supercapacitor energy. These supercapacitors deliver short-term high-power bursts into heater coils placed in the final half meter of pipe connecting the faucet.  During a period of less than a minute, 100--200~Wh of energy is released to the heater coil at a rate between 10--20~kW.


Based on the cost constraints of a commercial system and the regulatory authority requirements, a supercapacitor-only solution becomes prohibitively expensive. A review of current state-of-the-art energy storage systems show that no battery chemistry can withstand the associated charge-discharge cycles to reach the expected service life of 5--10 years. While developing the unique two-stage fast supercapacitor charger, a battery-supercapacitor hybrid system was developed for SCATMA as a commercially viable solution for rapid water heating.   
 
Dividing a supercapacitor bank into three parts and circulating them through a `charge-idle-discharge' sequence was already investigated for the surge resistant uninterrupted power supply. The effectiveness of this technique mandates fast supercapacitor charging. The proposed new charger achieves fast charging by using a high voltage source to overcome the five time constant charging time and a series coupled inductor to charge the  capacitor bank by dividing it into two parts. One part is charged using an over-voltaged dc source with capacitor bank terminal voltage monitoring and the other part is charged by the coupled-inductor using the energy stored in the inductor.

 





