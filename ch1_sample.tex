%-------------------------------------------------------------------
\chapter{\textbf{"Instantaneous" Water Heating}}
\chaptermark{ water heating}
%-------------------------------------------------------------------

\section{Scope}

The scope of this thesis is to present the results obtained in the PhD project ``SCATMA and supercapacitor-fast charger" undertaken by the author from March 2013 to January 2016. Most of the scientific findings of this project have been published in the form of peer reviewed conference and journal papers and a patent application. 

\section{Background and Motivation}

The inventions  developed by this group 
5
 include

\begin{itemize}
	
	\item SCALDO - SC-assisted low-dropout regulator - a high efficient dc--dc converter technique using low dropout regulators\cite{SCALDO_patent}
	\item SCASA - SC-assisted surge absorber - a technique developed to absorb power-line surges using SCs\cite{SCASA_patent}
	\item SRUPS - surge resistant uninterrupted power supply - a batteryless uninterrupted power supply~(UPS) technique capable of absorbing power line surges\cite{SRUPS:11}
	\item SCATMA - SC-assisted temperature modification apparatus - an ``instant" liquid flow heating technique based on stored energy\cite{SCATMA:14}
	\item SCAHDI - SC-assisted high density inverter - a SC technique to improve efficiency and packing density of a high power inverter
	
\end{itemize}

The initial SCATMA concept was developed in late 2012. 


\section{``Instantaneous'' Water Heating Problem}

Delayed hot water delivery in domestic water heating systems is a worldwide problem. A typical domestic hot water system will have a delay of around $5$--$30$~s on average depending on the distance of the tap from the central hot-water tank. 
There are a few commercial solutions proposed to address this problem. These solutions include 

\subsection{Requirements}

The above power and energy requirement is estimated for the worst case.

\begin{itemize}
	
	\item Electrical safety and isolation\\
	Due to safety concerns, a power source over 60 V cannot be utilized because of the low voltage requirements to prevent electrical shock. If mains power is used directly, galvanic isolation via a transformer supplying less than 60 V (rms) may be required as the high-power heater coil is directly in contact with the flowing water\cite{AS/NZSIWH04}.
	
	\item Domestic installation regulations\\
	Domestic subcircuits are protected using overcurrent protection devices\cite{AS/NZSRCD11(c)}. The maximum power rating for a domestic subcircuit will be 6.9 kW even if a 30 A dedicated sub-circuit is available. Therefore it is impractical for an existing electrical subcircuit to solely deliver this power.
	
\end{itemize}



...
% \pagestyle{empty}
% \begin{landscape}
% \begin{table}[htbp]
% \begin{center}
% \begin{tabular}{llll}
% ...
% \end{tabular}
% \end{center}
% \end{table} 
% \end{landscape}
% \pagestyle{plain}

\begin{sidewaystable}
  \centering
  \caption{A rotated table}
  	\begin{tabular}{clccc}
		\tableDline
		Parameters                            & minimum & average & maximum \\
		\tableline
		Water flow rate (litres/min)            & $4$   & $6$     & $8$  \\
		Delay in centrally heated hot water (s) & $5$   & $15$    & $30 $ \\ 
		Required temperature rise ($\degC$)     & $20$  & $30$    & $40$  \\ 
		Required energy (Wh)                    & $7.8$ & $52.5$  & $186.7$  \\
		Required power (kW)                     & $5.6$ & $12.6$  & $22.4$  \\  
			%
			\tableDline
		\end{tabular}
  \label{tab:test}
\end{sidewaystable}

\begin{table}[t]
	\centering
	\caption[Typical domestic ``instant"-water heating problem specifications]
	{Typical domestic ``instant"-water heating problem specifications}
	\label{ta:SCATMA_for_LWH}
	%
	\begin{tabular}{clccc}
		\tableDline
		Parameters                            & minimum & average & maximum \\
		\tableline
		Water flow rate (litres/min)            & $4$   & $6$     & $8$  \\
		Delay in centrally heated hot water (s) & $5$   & $15$    & $30 $ \\ 
		Required temperature rise ($\degC$)     & $20$  & $30$    & $40$  \\ 
		Required energy (Wh)                    & $7.8$ & $52.5$  & $186.7$  \\
		Required power (kW)                     & $5.6$ & $12.6$  & $22.4$  \\  
			%
			\tableDline
		\end{tabular}
	\end{table}














%-----------------------------------------------------------------------