%-------------------------------------------------------------------
\chapter{\textbf{Conclusions and Future Work}}
\label{ch:conclude_future}
%-------------------------------------------------------------------

\section{Summary and Conclusion}

The analysis on the ``instantaneous" water heating problem revealed the specifications for SCATMA.
\noindent Having identified pre-stored energy as the only solution to the problem, and the limitations of each device, the closest practical implementation is a hybridization of SCs with cheap lead-acid batteries. 


This method requires capacitor voltage monitoring to avoid overcharging. The converter works in transient mode and PFC is achieved without using any additional components~(by the same switch used in the primary side). The simulation and experimental results on the prototype design suggests

\begin{itemize}
	\item The equivalent charging rate can be calculated based on the primary and secondary peak currents
	\item The converter design has inherent transient mode power factor correction capability
	\item The design is energy limited based on the transformer design and component selection
\end{itemize}


\section{Future Work}

Battery Storage Optimisation:
o	The battery storage configuration requires more investigation. 

\noindent Supercapacitor Optimisation:
o	The current system is showing that a number of the specifications are being reached, or are close to being reached. Additional supercapacitors will be introduced to the system, to hit all targets (including worst case scenarios). 































